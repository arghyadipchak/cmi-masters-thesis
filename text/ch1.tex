Many-to-one matching with two-sided preferences has various real-world applications such as school choice. \ie placement of students to schools, college and university admission, hospital/residents programs, refugee resettlement or fair allocation in healthcare. In these scenarios, we are usually given two disjoint set of agents, $W$ and $F$, such that each agent has a preference list over some members of the other set, and each agent in $F$ has a \textit{capacity constraint} that limits the maximum number of agents on the other side it can be feasibly matched with. The
goal is to find a good matching (or assignment) between $W$ and $F$ without violating the capacity constraints. To unify the terminology, we consider the worker-firm allotment problem, and call the agents in $W$ the workers and the agents in $F$ the firms.

As to what defines a good matching, the answer varies from application to application. The simplest concept being that of a \textit{perfect} matching, which ensures that every worker is matched, achieving which can be crucial. The arguably most prominent and well-known concept however is that of \textit{stable} \cite{gale1962college, gusfield1989stable} matching, which ensures that no two agents form a blocking pair, \ie they do not prefer to be matched with each other over their assigned partners. Stability is a key desideratum and has been a standard criterion for many matching applications. Remarkably, for any given capacities, a stable matching of workers and firms always exists and can be computed using the celebrated \textit{deferred-acceptance} or \textit{Gale-Shapley} algorithm \cite{gale1962college, roth1984evolution}.

While the stable matching problem assumes fixed capacities, it is common to have flexible capacities in practice, particularly in settings with variable demand or popularity such as in vaccine distribution or course allocation. Flexibility refers to allowing the addition or removal of seats to firms that are either undersubscribed or oversubscribed, respectively. We will use the term \textit{capacity modification} to refer to change in the capacities of the firms by a central planner. The theoretical study of capacity modification was initiated by \Citeauthor{sonmez1997manipulation} \cite{sonmez1997manipulation}, who showed that under any stable matching algorithm, there exists a scenario where some firm is better off when its capacity is reduced.

Our focus of study will be the impact of \textit{capacity modification} on the set of stable matchings. Next, we will study how \textit{capacity modification} be done optimally to obtain stable and perfect matchings. Finally, we will study three other scenarios, with new parameters and constraints, and see how our initial approaches can be extended to solve the new problems.

\section{Thesis Outline}

The remainder of this thesis is organized as follows

\begin{description}
  \item[\autoref{chap:Preliminaries}] provides the preliminary background, definitions and fundamentals (like Gale-Shapley Algorithm and Rural Hospital's Theorem) for stable and many-to-one matching problems.
  \item[\autoref{chap:Impact-of-Capacity-Modification}] reviews \Citeauthor{gokhale2024capacity}'s \cite{gokhale2024capacity} work, focusing on the impact of \textit{capacity modification}. We study how the set of stable matchings respond to it.
  \item[\autoref{chap:Optimal-Capacity-Modification}] reviews \Citeauthor{chen2024optimal}'s \cite{chen2024optimal} work, focusing on the optimal capacity modification needed to obtain a stable and perfect matching. We study the computational complexity, parameterized complexity and approximability of the problem.
  \item[\autoref{chap:Our-Contribution}] answers the questions that arose while reviewing the last two chapters. We study scenarios with bound on capacity increase scale, varying costs for firms to increase capacity, and targeting only selected (of initially unmatched) workers to match.
\end{description}
